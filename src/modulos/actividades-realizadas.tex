\chapter{Actividades Realizadas}
	En el área de desarrollo de software colaboré en las distintas etapas del
	desarrollo del sistema que se desarrolló. Reuniones con usuarios para la
	obtención de requisitos, análisis de requisitos, diseño del sistema, desarrollo
	del sistema y por último puesta en producción del sistema.\\\
	
	A lo largo del periodo de prácticas utilice herrmientas que ayudan a ser
	productivos, tales son: Sistemas de control de versiones (Git, Bitbucket,
	Github, Gitlab), Entornos de Desarrollo Integrado (Eclipse, Netbeans, IntelliJ
	IDEA, Spring Tool Suite), editores de texto (Sublime Text, Atom, Visual Studio Code, Brackets), etc.
	
	\section{Obtención de requisitos}
		La obtención de requisitos es la etapa inicial y fundamental de cualquier tipo
		de proyecto de software que se quiera realizar. Se enfocan sólo en la visión
		del sistema que tiene el usuario. La funcionalidad del sistema, la interacción entre el
		usuario y el sistema, los errores que el sistema puede detectar y manejar son
		parte de los requisitos \cite{bernd1ingenieria}.
		
		La obtención de requerimientos incluye las siguientes actividades:
		
		\begin{itemize}
		  \item {\bf Identificación de actores.} {Durante esta actividad se
		  identifican los diferentes tipos de usuario que el sistema soportará.}
		  
		  \item {\bf Identificación de escenarios.} {Aquí se observan a los futuros
		  usuarios y se desarrollan un conjunto de escenarios posibles para las
		  distintas funcionalidades del sistema.}
		  
		  \item {\bf Identificación de casos de uso.} {Una vez de acuerdo el usuario
		  con el equipo de desarrollo, se abstraen los escenarios en casos de uso.}
		\end{itemize}
		
		Hubo reuniones con los usuarios, escuchándolos activamente, proponiendo ideas,
		debatiendo, descartando casos de uso innecesarios. Para llegar a un acuerdo.
		Dentro de esta labor el equipo tenía un panorama general de todo el sistema y a su vez
		en cada reunión se tenía ideas y modelos de funcionalidades listas para
		implementar; y presentarlos en la próxima reunión para ser aprobado por los
		interesados del proyecto.\\\
		
		Esta labor se me fue encomendada a medida que me iba familiarizando con el
		equipo, ya que es una parte crítica de un proyecto de software.
	
	\section{Análisis de requisitos}
		El análisis de requisitos se enfoca en la producción de un modelo del sistema.
		El análisis de requisitos le proporciona al diseñador del sistema una
		representación de información y función \cite{pressman06ingenieria}. Aunque puede ser que el
		modelo de análisis no sea comprensible para los usuarios, ayuda a que los
		diseñadores del sistema verifiquen la especificación del sistema producida
		durante la obtención de requisitos.
		
		El análisis de requisitos produce tres modelos individuales:
		
		\begin{itemize}
		  \item {\bf Modelo funcional.} {Representado por casos de uso y escenarios.}
		  
		  \item {\bf Modelo de objetos de análisis.} {Representado por diagramas de
		  clase y objetos.}
		  
		  \item {\bf Modelo dinámico.} {Representado por diagramas de estado y de
		  secuencia.}
		\end{itemize}
		
		Fui partícipe de esta actividad progresivamente, opinaba de acuerdo a los
		conocimientos que tenía, pero el equipo siempre estuvo dispuesto a ayudarme,
		con lo cual despejaba dudas e inquietudes.
		
	\section{Diseño de software}
		El Diseño de software es un proceso mediante el cual los requisitos se
		convierten en un plano para construir el software. Al inicio el plano
		representa una visión general del software. A medida que se va avanzando el
		proceso, se van representado partes del software a profundidad.\\\
		
		Parte del diseño de software se encarga de describir la descomposición en
		subsistemas desde el punto de vista de responsabilidades, dependencias, flujo
		de control, control de acceso y almacenamiento de datos.\\\
		
		Forme parte de esta labor activamente, ya que tenía un buen conocimiento
		acerca de patrones de diseño, bases de datos, descomposicion de sistemas, etc.
		
		
	
	\section{Desarrollo Frontend y Backend}
		La parte de programación, es donde utilizamos todos los diseños y diagramas
		para empezar a codificar en algún lenguaje de programación. El desarrollo
		frontend y backend requiere de habilidades y conceptos como: conocimientos
		de algoritmos, estructuras de datos, pruebas unitarias, pruebas end-to-end, integración
		continua, etc.\\\
		
		Ingresé a la empresa desarrollando únicamente en el lado del frontend,
		conforme avanzaba en conocimientos, comencé a desarrollar frontend y backend,
		los cuales me permitieron conocer las tecnologías actuales que dominan el mercado
		del desarrollo de software empresarial.
		
		
	
	\section{Despliegue de sistemas en la Nube (vps)}	
		En esta última actividad se pone en producción el sistema en su totalidad.
		Servicios como \textit{Google Cloud Platform, Digital Ocean, Amazon Web
		Services, etc}, requieren de habilidades de administración de servidores que
		ofrecen estas empresas. Distino a un hosting compartido, las plataformas como
		\textit{Google Cloud Platform} brindan un espacio de almacenamiento dedicado
		exclusivo y la libertad de configurar el servidor de acuerdo a necesidades
		específicas. También permite escalar a medida que el sistema crece.\\\
		
		Gracias a los conocimientos en servidores linux pude asumir esta tarea, la de
		configurar, instalar paquetes y dejar todo listo para el funcionamiento
		correcto en la nube.\\\
		
		\textit{Conforme el proyecto avanzaba, pude desenvolverme adecuadamente en
		todas las actividades mencionadas.}
