\chapter{Descripción del Proyecto Realizado}
	Durante el periodo de prácticas se desarrolló un Sistema de Salud Ocupacional
	(basado en web), realizando un trabajo en equipo; por lo que el practicante
	estaba a cargo del desarrollo de ciertas funcionalidades del proyecto.
	
	\section{Objetivo}
		Análisis, Diseño e Implementación de un Sistema de Salud Ocupacional para la
		Clínica Del Valle de la ciudad de Juliaca.
		
	\section{Justificación}
		El sistema es necesario para la Clínica del Valle, porque les permitirá
		ahorrar tiempo y recursos, como también reducirá la tasa de errores; es necesario para
		los pacientes, porque se les brindará un servicio más rápido y con resultados
		precisos.
		
	\section{Planificación}
		La construcción del sistema tuvo una duración aproximada de seis meses. La
		planificación era flexible en cuanto a reuniones (cada dos semanas) con el
		cliente. Se llevó a cabo una primera reunión con el cliente, quién
		requería las funcionalidades primordiales/urgentes del sistema. Pasada las dos
		semanas se tenia un producto mínimo viable y funcional. Las iteraciones nos
		permitía ír añadiendo más funcionalidades.
		
	\section{Metodología}
		Se usó una metodología ágil apoyado en el conjunto de ideas que nos brinda
		``kanban'' y el Análisis y Diseño Orientado a Objetos.