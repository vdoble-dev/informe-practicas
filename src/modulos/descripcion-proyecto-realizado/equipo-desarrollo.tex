\section{Equipo de Desarrollo}
	\subsection{Gerente de Proyecto}
		\begin{itemize}
			\item El Gerente de proyecto coordina con el usuario líder
			\item Monitorea el desempeño de los recursos y tomara acciones correctivas
			\item Mantiene informado al usuario líder, actúa como enlace entre el
			personal del proyecto y otros tomadores de decisiones
		\end{itemize}
		
	\subsection{Jefe de Proyecto}
		
		El jefe de proyecto asigna los recursos, gestiona las prioridades, coordina las
		interacciones con los clientes y usuarios, y mantiene al equipo del proyecto
		enfocado en los objetivos. El jefe de proyecto también establece un conjunto de
		prácticas que aseguran la integridad y calidad de los artefactos del proyecto.
		Además, el jefe de proyecto se encargará de supervisar el establecimiento de la
		arquitectura del sistema. Gestión de riesgos. Planificación y control del
		proyecto.\\\
		
		El Jefe de Proyecto tiene las siguientes características:
		
		\begin{itemize}
			\item Experiencia en el diseño y desarrollo de aplicaciones Internet
			(Intranet/Extranet)
			
			\item Experiencia Liderando proyectos Internet
			
			\item Dominio de las metodologías más eficaces para el análisis de la
			información
			
			\item Ha participado en distintos proyectos dirigiendo y cumpliendo otro
			tipo de perfiles lo que hacen de él un elemento con la capacidad de
			distribuir las tareas de un proyecto
			
			\item Posee también, conocimientos especializados en áreas específicas del
			proyecto como puede ser: diseño, implantación y administración de Bases de
			Datos, servidores Web
		\end{itemize}
	
	\subsection{Analista de Sistemas}
		El Analista de Sistemas es la persona encargada de la captura, especificación y
		validación de requisitos, interactuando con el cliente y los usuarios mediante
		entrevistas. Elabora el Modelo de Análisis y Diseño. Colabora en la elaboración
		de las pruebas funcionales y el modelo de datos.
		
	\subsection{Arquitecto de Software}
		El Arquitecto es el encargado de definir el esquema de trabajo, la
		tecnología, gestión de requisitos, gestión de configuración e impactoen los
		cambios.
		
	\subsection{Analista Programador}
		El Analista Programador está encargado de la construcción de prototipos,
		desarrollo de los componentes. Colabora en la elaboración de las pruebas
		funcionales y en las validaciones con el usuario.\\\
		
		Dentro de este perfil tenemos los siguientes:
		
		\subsubsection{Analista Programador Junior}
			\begin{itemize}
				\item Egresado de la Universidad
				\item Conocimientos en Tecnología JAVA
				\item Conocimientos en Desarrollo JAVA
				\item Capacidad para trabajar en equipo
			\end{itemize}
			
		\subsubsection{Analista Programador Estándar}
			\begin{itemize}
				\item Egresado de la Universidad
				\item Experiencia en Análisis, Diseño y Desarrollo
				\item Conocimientos en Tecnología JAVA
				\item Experiencia en desarrollo JAVA (2 años)
				\item Capacidad para trabajar en equipo
			\end{itemize}
			
		\subsubsection{Analista Programador Senior}
			\begin{itemize}
				\item Egresado de la Universidad
				\item Experiencia en Análisis, Diseño y Desarrollo
				\item Conocimientos en Tecnología JAVA
				\item Conocimientos de arquitecturas
				\item Conocimiento de frameworks
				\item Experiencia en desarrollo JAVA (+4 años)
				\item Capacidad para trabajar en equipo
			\end{itemize}		
	
	\subsection{Programador}
		El programador es aquel que posee una amplia experiencia en desarrollo de
		aplicaciones Internet. Posee un criterio amplio y experiencia necesaria para
		el desarrollo de las aplicaciones del proyecto, de manera que resulten ser
		óptimas en tiempo de ejecución y uso de recursos.\\\
		
		Se encarga de investigar aquellos puntos del proyecto que necesiten una mejor
		visión, de manera que encuentre la mejor alternativa de solución que colabore
		con un óptimo desarrollo del proyecto. Tiene una base sólida en el
		conocimiento de sistemas operativos, herramientas de conectividad y
		desarrollo sobre distintas bases de datos, lo que le permite cumplir con los
		requerimientos del proyecto de manera eficaz.\\\
		
		Posee el conocimiento de las últimas herramientas de desarrollo en Internet,
		de manera que pueda utilizarlas eficientemente durante el desarrollo de sus
		aplicaciones.
		
	\subsection{Administrador de Bases de Datos}
		El Administrador de Base de Datos se encarga de autorizar el acceso a
		la base de datos, de coordinar y vigilar su empleo, se encarga de la
		instalación y configuración de los motores de Base de datos. Realiza además
		las siguientes funciones:
		
		\begin{itemize}
			\item Mejoras a los motores de Base de Datos.
							
			\item Manejo de seguridad: a nivel de Base de Datos definiendo la estructura y
			sus objetos (tablas, campos, procedimientos, usuarios, roles, triggers, etc.), y
			a nivel de usuarios definiendo el acceso de los diferentes usuarios a los
			objetos de la Base de Datos
			
			\item Monitoreo del espacio físico y los archivos lógicos para optimizar la Base
			de Datos
			
			\item Respaldos y Recuperación de Información
			\item Transferir e Importar Datos y estructuras de Base de Datos externas
			\item Mantenimiento a la Base de Datos
			\item Implementación de Réplicas
		\end{itemize}
		
	\subsection{Diseñador}
		Diseña la interfaz de la aplicación, trabaja desde el análisis para tener una
		visión integral del proyecto. Analiza la situación, define la estrategia de 	
		comunicación, la estructura lógica del site, la arquitectura de la
		información, jerarquiza los contenidos de acuerdo a su importancia. Define el
		estilo gráfico y de sus principales páginas aprovechando al máximo todas las
		herramientas de comunicación en Internet.
		
	\subsection{Documentador}
		Elabora todos los documentos requeridos para el sistema.